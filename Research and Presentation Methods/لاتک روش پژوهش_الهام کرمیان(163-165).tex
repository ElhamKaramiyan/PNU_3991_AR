\documentclass{article}
\usepackage{multicol}
\usepackage{xcolor}
\usepackage{graphicx}
\linespread{1.35}
\usepackage{amsmath}
\usepackage{color}
\usepackage{tikz}
\usetikzlibrary{arrows,automata}

\begin{document}

\begin{flushright}
 \texttt{SURVEYS} \hspace*{1cm} \textbf{163}
\end{flushright}

\vspace*{0.5cm}
of respondents will be using mailers that support HTML and thus the enhanced formatting features will normally make this the solution of choice.\\
\begin{itemize}
  \item send long surveys in two four parts with one part sent week. Respondents are more likely to give four small pieces of their time rather than one long com-ponent. In addition, once a respondent has completed one component of the survey, they will likely feel more committed to seeing the task through and com-pleting subsequent sections.\\
\end{itemize}

\vspace*{0.4cm}
\large{
\textbf{Web-Based Surveys}
}

\vspace*{0.2cm}
Web-based survevs have a number of advantages over email survevs. However the creation and administration of Web-based surveys are considerably more expensive and complex than email surveys. Web surveys are stored on an active Web server and thus are able to take advantage of the processing power of the server or the processing power of the respondent's machine to validate the survey while it is being completed. Validation is extremely useful as missed responses, outrageous responses, and incon-sistencies can be analyzed instantly and the respondent can be asked to correct obvious errors. Without such correction, the researcher might be forced to discard the survey, thus reducing the effective return rate.\\

\hspace*{0.5cm} The processing power of the server can also be used to provide transparent branching, so that the answer from one question allows subsequent questions to be customized. Customization avoids the complexity and potential confusion of paper and email surveys that provide instructions such as: ''If answer is yes, then go to ques-tion \#10, else go to question \#12.'' Web-based surveys can be created as a single page that scrolls to reveal the full survey. Alternatively, each question can be formatted as a separate Web page, with the response from each question triggering the delivery of subsequent questions. There seems to be no clear advantage to either approach, though each has advantages and disadvantages. Longer forms allow the respondent to gauge the number of questions in full survey and permit scrolling backwards to review or even change previous answers. Error checking on long forms must be built using more complex client-side programming. Client-side programming may not be available in all versions of participant browsers and may be disabled by other partici-pants. Short forms allow for branching and immediate server-side error checking but may be frustrating to participants if the connection speed to the server is slow.\\ 

Thus, the degree of complexity of the from and the planned connection speed of typ-ical respondents help determine how long to make individual components of a Web-based survey.\\
\hspace*{0.5cm} In an interesting overview of the mechanics (down to illustrations of actual cod-ing) of client- and server-side processing of Web surveys, White, Carey, and Dailey (2001) illustrate actual code (http://fcit.coedu.usf.edu/surveydemo/) and discuss their use of a Web survey to determine students' potential for success in distance education courses. They found that coding for automatic tabulation of results was easier now than some years ago but still presents challenges for e-researchers who do not know how to program. Thus, their findings reinforce the efficacy of using commercial pack-ages as described later in this chapter.\\

\newpage

\begin{flushleft}
\textbf{164}\hspace*{1cm} \texttt{CHAPTER ELEVEN}
\end{flushleft}

\vspace*{0.5cm}
\hspace*{0.5cm} Since the Web$-$based survey is live on the Web, many of the more advanced graphical and annotation features of the Web can be used. Color and font size can be used to draw attention to an important component; however, the excessive use of either can be distracting. Any of the paper or online books on Web page and site design should be consulted to provide guidelines for e-researchers who are designing their own survey interface. The Yale style guide available at $http://info.med.yale.edu/caim/manual/contents.html$ provides a free and very useful resource for Web page creators.\\
\hspace*{0.5cm} Very sophisticated multimedia surveys can also be created in Java and other Web$-$based programming languages. The developers of such surveys claim increased usage because of the attractiveness and instructiveness of the format, however we have not seen empirical evidence to back such claims. One commercial company, Survey$-$Said $(http://www.surveysaid.com)$, provides examples of both Web$-$based and multi$-$media, Java$-$enabled surveys at $http://www.surveysaid.com/marketing_masters/ssdocs/examplws.htm$.\\

 \hspace*{0.5cm} Multipage surveys produced by the Web server can also be used to determine the length of time that a respondent spends on each page (or individual question) thus allowing the researcher greater insight into the respondent's behavior during the sur-vey completion. Batagelgj and Vehovar (1998) found that completion rates for single (long) page Web-based surveys were not significantly different from multipage ver-sions, although, again, multipage versions took longer to complete. Given the advan-tages of skipping, jumping, and time analysis provided by multipage surveys, it seems they are the preferred format except when connection speeds are known to be slow or problematic.\\
\hspace*{0.5cm} A good format consists of clear and brief instructions, transitional phrases, coher-ent groupings of items, appropriately used graphics, and an aesthetically pleasing arrangement of questions. The general principle to follow is to put the need of the respondent first. Put yourself in the respondent's place and you can come up with cre-ative ways to make the questionnaire appealing to the eye and as possible to answer.\\

\vspace*{1cm}
\large{
\textbf{TIPS FOR WEB-BASED SURVEYS}\\
}

\vspace*{0.3cm}
\begin{itemize}
  \item Set metatags in your Web-based survey so that search engines can troll and prop-erly index your site if you wish to attract as many participants as possible.\\
  \item Include a direct link to the Web-based survey in any invitational email.\\
  \item Include a cut-off date to both motivate respondents and give yourself a date for analysis.\\
  \item Provide clues as to the total length of the survey. This is especially important with multipart Web surveys since the respondent may not be able to quickly scan to the end of the survey to see how much more time is required. A graphical meter (like a gas gauge) indicating progress through the full survey works well for this purpose.\\
  \item Be thorough when debugging, pilot testing, and error checking to insure that respondents are not prohibited from completing the survey due to a misunder-standing or misreading of the survey.\\
\end{itemize}

\newpage
\begin{flushright}
 \texttt{SURVEYS} \hspace*{1cm} \textbf{165}
\end{flushright}

\vspace*{0.5cm}
\begin{itemize}
  \item Use automated checking software to insure that all internal and external links are active and that the site is accessible to handicapped users. See $http://www.help4web.net$ for links to testing sites and other Web creation resources.\\
  \item Concerns with multiple submissions by a single user can be reduced through strategies that make use of cookies or that assign unique passwords for each user. Cookies are code that is attached to the users' browser when they interact with a Web site. They can be used effectively to remember who the users sre or to remember where the users were in the process of completing a Web survey.However, some users intentionally turn off the browser's capacity to store cook-ise, and thus cookie-based strategies may fail for certain respondents. Unique identifiers such as passwords may be included in the invitational email. This strategy can reduce anonymity and convenience but greatly enhances the respon-dent's flexibility in that answers can be reviewed or the survey completed over multiple sessions, because the Web server knows what particular information has been keyed by invidual respondents.\\
\end{itemize}

\begin{itemize}
  \item Do not attempt to force participants to respond to every question through use of server- or client-side validation routines. Participants may have moral or intel-lectual reasons for refusing to answer a question. Forcing an answer before dis-plying    subsequent   questions  will  only  result  in  reluctant  participants abandoning the survey. Therefore, for ethical and practical reasons participants should be allowed to indicate they have no answer to any particular question and be allowed to continue.\\
\end{itemize}

\vspace*{0.5cm}
\large{
\textbf{Overcoming Sample Bias in Web-Based Surveys}\\
}

\vspace*{0.1cm}
One of the inherent problems of Web-based surveys is the self-selection that occurs among respondents. As mentioned, a survey is desiged to inform us about the full population, not just those who are inclined to complete surveys. This problem is espe-cially challenging with Web-based surveys, as return rates can be very low, especially when the invitation to participate is provided only through a passive link on a Web page.
Further, it is almost impossible to calculate the response rate from this form of invitation. Have all those who have seen the page actually made a decision to partici-pate or not in the survey? Or have many merely skimmed over the link or banner as is customary with much Vet advertising?\\

The Swedish researcher, Micael Dahlen, proposed a method of sampling for Web-based surveys that illustrates an innovative way to control the gathering of a Web sample (Dahlen, 1998 ). Frist, the researcher carefully defines the sample frame, often selecting those who frequent a particular site on Web. Second, a random selection is made from this sample frame and a specific invitation is made to this sam-ple. Dahlen experimented with a server-side javascript to create a popup window to issue the invitation, but there are a variety of alternative programming techniques available on most large-scale sites that can accomplish this task. The respondent is thus challenged to participate in the survey and decision to participate or not allows for the calculation of response rate. Finally, Dahlen recommends a means to identify the respondents so that the invitation is not issued to return visitors to the site. Dahlen suggests the use of browser-stored cookies for this task; however, as discussed earlier,\\

\end{document} 
